%%% -*-LaTeX-*-
%%% reversiblemarkovchains.tex.orig
%%% Prettyprinted by texpretty lex version 0.02 [21-May-2001]
%%% on Fri Aug 26 08:03:44 2022
%%% for Steve Dunbar (sdunbar@family-desktop)

\documentclass[12pt]{article}

\input{../../../../etc/macros} %\input{../../../../etc/mzlatex_macros}
\input{../../../../etc/pdf_macros}

\bibliographystyle{plain}

\begin{document}

\myheader \mytitle

\hr

\sectiontitle{Reversible Markov Chains}

\hr

\usefirefox

% \hr

% \visual{Study Tip}{../../../../CommonInformation/Lessons/studytip.png}
% \section*{Study Tip}

\hr

\visual{Rating}{../../../../CommonInformation/Lessons/rating.png}
\section*{Rating} %one of
% Everyone: contains no mathematics.
% Student: contains scenes of mild algebra or calculus that may require guidance.
Mathematically Mature:  may contain mathematics beyond calculus with
proofs.  % Mathematicians Only: prolonged scenes of intense rigor.

\hr

\visual{Section Starter Question}{../../../../CommonInformation/Lessons/question_mark.png}
\section*{Section Starter Question}

Imagine a video is made of the Ehrenfest urn model started in its
stationary distribution.  Would it be possible to distinguish whether
the video is played forward or in reverse simply by viewing the video
and observing the movement of the balls between the urns?

\hr

\visual{Key Concepts}{../../../../CommonInformation/Lessons/keyconcepts.png}
\section*{Key Concepts}

\begin{enumerate}
    \item
        A Markov chain with invariant distribution \( \pi \) is
        reversible if and only if the state space has a probability
        distribution \( \pi \) such that
        \[
            \pi_i P_{ij} = \pi_j P_{ji}
        \] for all \( i \) and \( j \) and \( \pi \) is the unique
        stationary distribution.
    \item
        The condition says that starting from the stationary
        distribution
        \[
            \Prob{X_n =i, X_{n+1} = j} = \Prob{X_n =j, X_{n+1} = i}
        \] so the chain transitions from \( i \) to \( j \) as often as
        it transitions from \( j \) to \( i \).
    \item
        Assume the Markov chain is irreducible and positive recurrent
        and probability distribution \( \pi \) satisfies the detailed
        balance equation
        \[
            \pi_i P_{ij} = \pi_j P_{ji}.
        \] Then the distribution \( \pi \) is the stationary
        distribution of the chain and the chain is reversible.
    \item
        Let \( P \) be a \( k \times k \) transition matrix to which an
        Algorithm of row and column operations is applied, resulting in \(
        P^{\star} \).  Then \( P \) is reversible if and only if \( P^{\star}
        \) is symmetric.
\end{enumerate}

\hr

\visual{Vocabulary}{../../../../CommonInformation/Lessons/vocabulary.png}
\section*{Vocabulary}
\begin{enumerate}
    \item
        Assume \( X \) is an irreducible and positive recurrent chain,
        started at its unique invariant distribution \( \pi \).  This
        means \( X_0 \) has distribution \( \pi \), and all other \( X_n
        \) have distribution \( \pi \) as well.  Now suppose that for
        every \( n \), the sequence of random variables \( X_1, X_2,
        \dots , X_n \) has the same joint distribution as the
        time-reversed sequence \( X_n, X_{n-1}, \dots , X_1 \), then the
        chain \( X \) is \defn{reversible}.
    \item
        The condition
        \[
            \pi_i P_{ij} = \pi_j P_{ji}
        \] is the \defn{detailed balance equation}.
    \item
        \defn{Kolmogorov's loop criterion} is
        \[
            P_{j_0 j_1} P_{j_1 j_2} \cdots P_{j_{k-1} j_k} P_{j_k j_0} =
            P_{j_0 j_k} P_{j_k j_{k-1}} \cdots P_{j_2 j_1} P_{j_1 j_0}
        \] for every finite sequence of distinct states \( j_0, j_1,
        \dots, j_k \).
\end{enumerate}

\section*{Notation}
\begin{enumerate}
    \item
        \( X \), \( X_0 \), \( X_n \) -- Markov chain, starting state,
        general step of the chain.
    \item
        \( k \) -- Number of states in the Markov chain.
    \item
        \( \pi \), \( \pi_i \) -- Stationary distribution and general
        element of the stationary distribution.
    \item
        \( i \), \( j \), \( i_{\ell} \) etc.  -- arbitrary states of
        the Markov chain.
    \item
        \( P \), \( P_{ij} \) -- Probability transition matrix and
        general element of probability transition matrix.
    \item
        \( P^{\star} \) -- the matrix resulting from a row (or column)
        multiplication operation on \( P \).
\end{enumerate}

\hr

\visual{Mathematical Ideas}{../../../../CommonInformation/Lessons/mathematicalideas.png}
\section*{Mathematical Ideas} % \subsection*{Reversibility}

Assume \( X \) is an irreducible and positive recurrent chain, started
at its unique invariant distribution \( \pi \).  This means \( X_0 \)
has distribution \( \pi \), and all successive \( X_n \) have
distribution \( \pi \) as well.  Now suppose that, for every \( n \),
the sequence of random variables \( X_1, X_2, \dots , X_n \) has the
same joint distribution as the time-reversed sequence \( X_n, X_{n-1},
\dots , X_1 \).  Then the chain \( X \) is \defn{reversible}%
\index{reversible}%
\index{Markov chain!reversible}
and its invariant distribution \( \pi \) is also said to be reversible.
This means a recorded simulation of a reversible chain looks the same if
the ``video'' is run backwards.

\subsection*{Detailed Balance Equations}

\begin{proposition}
    If a Markov chain is started in \( \pi \), then the time-reversed
    chain has the Markov property.
\end{proposition}

\begin{proof}
    \begin{align*}
        & \Prob{X_\ell = i \given X_{\ell+1} = j, X_{\ell+2} = i_{\ell+2},
        \dots, X_n = i_n} \\
        &\qquad = \frac{\Prob{X_\ell = i, X_{\ell+1} = j, X_{\ell+2} = i_
        {\ell+2}, \dots, X_n = i_n}}{\Prob{X_{\ell+1} = j, X_{\ell+2} =
        i_{\ell+2}, \dots, X_n}} \\
        &\qquad = \frac{\pi_i P_{ij}P_{j \ell+2}\cdots P_{i_{n-1} i_n}}{\pi_j
        P_{j \ell+2}\cdots P_{i_{n-1} i_n}} \\
        &\qquad = \frac{\pi_i P_{ij}}{\pi_j}
    \end{align*}
    which depends only on \( i \) and \( j \), so the time-reversed
    chain has the Markov property.
\end{proof}

\begin{remark}
    The definition is inconvenient because it requires the stationary
    distribution \( \pi \) in advance to check if the chain is
    time-reversible.  The following theorem avoids having to know \( \pi
    \) in advance and can even help find \( \pi \), see Theorem~%
    \ref{thm:reversiblemarkovchain:detbaleqn}.

\end{remark}
\begin{theorem}[Reversibility Condition]
    A Markov chain is reversible if and only if the state space has a
    probability distribution \( \pi \) such that
    \[
        \pi_i P_{ij} = \pi_j P_{ji}
    \] for all \( i \) and \( j \) and then \( \pi \) is the unique
    stationary distribution.
\end{theorem}

\begin{remark}
    The condition
    \[
        \pi_i P_{ij} = \pi_j P_{ji}
    \] is the \defn{detailed balance equation}.%
    \index{detailed balance equation}
\end{remark}

\begin{proof}
    \begin{description}
        \item[(\(\Rightarrow\))]
            From the proof of the Proposition, the time-reversed
            sequence of random variables satisfies
            \[
                \Prob{X_\ell = i \given X_{\ell+1} = j} = \frac{\pi_i P_
                {ij}}{\pi_j}.
            \] If the Markov chain is reversible, the previous
            expression must be the same as the forward transition
            probability \( \Prob{X_{\ell+1} = i \given X_{\ell} = j} = P_
            {ji} \).  Then the detailed balance equation holds.
        \item[(\(\Leftarrow\))]
            From the proposition, the time-reversed chain is Markov.  If
            the original and the time-reversed chain both start at the
            same invariant distribution and satisfy the detailed
            balance, then the transition probabilities are the same
            forward or backward.  Then by definition the chain is
            reversible.
    \end{description}
\end{proof}

\begin{remark}
    The condition says that starting from the stationary distribution
    \[
        \Prob{X_n =i, X_{n+1} = j} = \Prob{X_n =j, X_{n+1} = i}
    \] so the chain transitions from \( i \) to \( j \) as often as it
    transitions from \( j \) to \( i \).
\end{remark}

\begin{example}
    If the \( k \times k \) transition probability matrix is symmetric,
    then \( \pi = (1/k, \dots, 1/k) \) satisfies the detailed balance
    equation and the corresponding Markov chain is reversible.
\end{example}

\begin{remark}
    If there exist \( 2 \) states such that \( P_{ij} > 0 \) but \( P_{ji}
    = 0 \) then the detailed balance equation fails, since the limiting
    probabilities of all entries of an irreducible and positive
    recurrent chain are nonzero.  Then the chain is not reversible.
\end{remark}

\begin{theorem}
    \label{thm:reversiblemarkovchain:detbaleqn} Assume the Markov chain is
    irreducible and positive recurrent and probability distribution \(
    \pi \) satisfies the detailed balance equations
    \[
        \pi_i P_{ij} = \pi_j P_{ji}
    \]
    \begin{enumerate}
        \item
            The distribution \( \pi \) is the stationary distribution of
            the chain.
        \item
            The chain is reversible.
    \end{enumerate}
\end{theorem}

\begin{proof}
    Using the detailed balance equations
    \[
        \sum_j \pi_j P_{ji} = \sum_j \pi_i P_{ij} = \pi_i \sum_j P_{ij}
        = \pi_i.
    \] Since the chains is irreducible and positive recurrent, then \(
    \pi \) must be the unique stationary distribution.
\end{proof}

\subsection*{Kolmogorov's Loop Criterion}

A transition probability matrix is \emph{reversible}%
\index{reversible matrix}
if the corresponding Markov chain is reversible.
\begin{definition}
    The equality condition
    \begin{equation}
        \label{eq:reversiblemarkovchain:loopprob} P_{j_0 j_1} P_{j_1 j_2} \cdots
        P_{j_{\ell-1} j_\ell} P_{j_\ell j_0} = P_{j_0 j_\ell} P_{j_\ell
        j_{\ell-1}} \cdots P_{j_2 j_1} P_{j_1 j_0}
    \end{equation}
    is called \defn{Kolmogorov's loop criterion}.%
    \index{Kolmogorov's loop criterion}
\end{definition}

\begin{theorem}
    An irreducible and positive recurrent Markov chain is reversible if
    and only if
    \[
        P_{j_0 j_1} P_{j_1 j_2} \cdots P_{j_{\ell-1} j_\ell} P_{j_\ell
        j_0} = P_{j_0 j_\ell} P_{j_\ell j_{\ell-1}} \cdots P_{j_2 j_1} P_
        {j_1 j_0}
    \] for every finite sequence of distinct states \( j_0, j_1, \dots, j_\ell
    \).
\end{theorem}

\begin{proof}
    \begin{description}
        \item[(\(\Rightarrow\))]
            An irreducible and positive recurrent Markov chain has a
            stationary distribution \( \pi \).  Since the matrix is
            reversible \( P_{ij} = \frac{\pi_j P_{ji}}{\pi_i} \) for
            every \( i \) and \( j \).  Substituting this relation for
            every probability on the left side of the criterion
            \begin{multline*}
                P_{j_0 j_1} P_{j_1,j_2} \cdots P_{j_{\ell-1} j_\ell} P_{j_\ell,
                j_0} = \\
                \frac{\pi_{j_1} P_{j_1 j_0}}{\pi_{j_0}} \cdot \frac{\pi_
                {j_2} P_{j_2 j_1}}{\pi_{j_1}} \cdots \frac{\pi_{j_{\ell}}
                P_{j_\ell j_{\ell-1}}}{\pi_{j_{\ell-1}}} \cdots \frac{\pi_{j_0}
                P_{j_0 j_\ell}}{\pi_{j_{\ell}}}.
            \end{multline*}
            Canceling the \( \pi_i \) factors and reversing the order of
            multiplication gives the right side of the Kolmogorov loop
            criterion.
        \item[(\(\Leftarrow\))]
            Assume the loop criterion holds.  Fix states \( j_{\ell} \) and \(
            j_{0} \).  Then
            \begin{multline*}
                \Prob{X_n = j_{\ell}, X_{n-1} = j_{\ell_{n-1}}, \dots, X_2 = j_2, X_1
                = j_1 \given X_0 =j_0} =\\
                P_{j_0 j_1} P_{j_1 j_2} \cdots P_{j_{n-1} j_{\ell}} =
                \frac{P_{j_{0} j_{\ell}}}{P_{j_{\ell} j_{0}}}
                P_{j_{n-2} j_{n-3}} \cdots P_{j_3
                j_2} P_{j_2 j_1} P_{j_1 j_0}= \\
              \frac{P_{j_{0} j_{\ell}}}{P_{j_{\ell} j_{0}}}
              \Prob{X_n = j_0, X_{n-1} = j_1,
                \dots, X_2 = j_{n-2}, X_1 = j_{n-1} \given X_0 = j_{\ell}}
            \end{multline*}
            where the second equality is a rearrangment of the loop criterion.
            Now sum both sides over all choices of the intermediate
            states \( j_1 \), \( j_2 \), \dots, \( j_{\ell_{n-1}} \) to obtain
            the following relation among the \( n \) step transition
            probabilities
            \[
                P_{j_0 j_{\ell}}^{n} = \frac{P_{j_{0}
                    j_{\ell}}}{P_{j_{\ell} j_{0}}} P_{j_{\ell} j_0}^{n}.
            \] Multiply by \( P_{j_{\ell} j_{0}} \) and take the limit in \( n \) on
            both sides and use that the limits of the multi-step
            transition probabilities are the stationary probabilities,
            guaranteed to exist since the Markov chain is irreducible
            and positive recurrent.  The result is
            \[
                \pi_{j_0} P_{j_0 j_{\ell}} = \pi_{j_{\ell}}
                P_{j_{\ell} j_0}
            \] which is exactly the reversibility criterion.
    \end{description}
\end{proof}

\begin{remark}
    In words, Kolmogorov's loop criterion says a Markov transition
    matrix is reversible if and only for every loop of distinct states,
    the forward loop probability product equals the backward loop
    probability product.
\end{remark}

For a two-state Markov chain Kolmogorov's loop criterion is always
satisfied since \( P_{12} P_{21} = P_{12} P_{21} \).  If the transition
matrix is symmetric, then \( P_{ij} = P_{ji} \) for all \( i \), \( j \),
so Kolmogorov's loop criterion is always satisfied and the chain is
reversible.

Usually loop checking involves much computational work.  The obvious
reason is that the number of loops that need to be checked grows very
quickly with \( n \) where \( n \) is the number of states.  See Table~%
\ref{tab:reversiblemarkovchains:loopeqns}.  The following Proposition
counts the number of equations that must be checked to apply
Kolmogorov's loop criterion.

\begin{proposition}
    For an \( k \) state Markov chain, with \( k \ge 3 \), the number of
    equations that must be checked for reversibility by Kolmogorov's
    method is
    \[
        \sum_{\nu=3}^n \binom{n}{\nu} \frac{(\nu-1)!}{2}.
    \]
\end{proposition}

\begin{proof}
    \begin{enumerate}
        \item
            For a \( 3 \)-state Markov chain, only \( 1 \) equation
            \[
                P_{12}P_{23}P_{31} = P_{31} P_{32} P_{21}
            \] needs checking, since all \( 2 \) step loops
            automatically satisfy the criterion and any other length \(
            3 \) loop over the same states results in the same equation.
        \item
            For \( n=4 \), all \( 2 \) step loops automatically satisfy
            the criterion.  Next check each loop of \( 3 \) states. For \(
            3 \) state loops, choose any \( 3 \) of \( 4 \) states and
            there is one equation for each.  Finally, for each loop of \(
            4 \) states, fix the starting state. Then the other \( 3 \)
            states have \( 3! \) possible orders. However, the other
            side of the equation is just the reversed path, so there are
            only \( \frac{3!}{2} \) paths involving \( 4 \) states with
            the first state fixed.  In total, the number of loop
            equations is
            \[
                \binom{4}{3} + \binom{4}{4} \frac{(4-1)!}{2} = 7.
            \]
        \item
            The previous argument easily generalizes to larger values of
            \( n \).
    \end{enumerate}
\end{proof}

\begin{table}
    \centering
    \begin{tabular}{ccccccccccc}
        \(k\)       & 1 & 2 & 3 & 4 & 5  & 6   & 7       & 8       & 9        & 10        \\ 
        equations & 0 & 0 & 1 & 7 & 37 & 197 & 1{,}172 & 8{,}018 & 62{,}814 & 556{,}014
    \end{tabular}
    \caption{Number of equation to be checked for a Markov chain with \(
    k \) states.}%
    \label{tab:reversiblemarkovchains:loopeqns}
\end{table}

Next define operations to transform transition matrices in such a way as
to preserve their reversibility status (either reversible or
non-reversible).  These transformations will be useful in creating new
reversible Markov chains from existing ones, and for checking
reversibility of Markov chains.  A \emph{row multiplication operation}
on row \( i \) of a Markov transition matrix is the multiplication of
row \( i \) by a positive constant leaving the sum of the non-diagonal
elements at most \( 1 \), followed by an adjustment to \( p_{ii} \) to
make the row sum exactly to \( 1 \).  A \emph{column multiplication
operation} on column \( j \) of a Markov transition matrix is the
multiplication of column \( j \) by a positive constant of allowable
size (so no row sums exceed 1) followed by adjustments to all diagonal
entries to make every row sum exactly 1.

\begin{lemma}
    A Markov chain matrix P maintains its reversibility status after a
    row multiplication operation or a column multiplication operation.
\end{lemma}

\begin{proof}
    \begin{enumerate}
        \item
            The Kolmogorov loop criterion states that \( P \) is
            reversible if and only if for all loops of distinct states,
            the forward loop probability product equals the backward
            loop probability product.
        \item
            Let the \( i \)th row of \( P \) correspond to state \( i \).
            If a loop does not include state \( i \), then a
            multiplication row operation on row \( i \) has no effect on
            the forward and backward loop products.
        \item
            Otherwise, state \( i \) appears in the first subscript of a
            forward loop probability if and only if it appears in the
            first subscript in the corresponding adjacent transition
            probability in the backward loop probability.  See equation
            \eqref{eq:reversiblemarkovchain:loopprob}.
        \item
            So the row operation will have an identical effect on both
            sides of the loop product.
        \item
            A similar conclusion holds for column product operations.
    \end{enumerate}
\end{proof}

\begin{remark}
    Let \( P^{\star} \) be the matrix resulting from a row (or column)
    multiplication operation on \( P \), then the limiting probabilities
    for \( P^{\star} \) in the lemma are generally different from the
    limiting probabilities for \( P \).
\end{remark}

If the matrix \( P \) is an \( k \times k \) probability transition
matrix, then at most \( k - 1 \) row or column multiplication operations
will determine if \( P \) is reversible.

\begin{algorithm}[H]
  \DontPrintSemicolon
  \KwData{Transition probability matrix \( P \)}
  \KwResult{Symmetric  matrix \( P^{\star} \) }
    Pick two symmetric positions in \( P \) with non-zero entries, say
    \( P_{i_1 i_2}
    \) and \( P_{i_2 i_1} \).\;
    \( S \leftarrow \set{i_1  i_2} \).\;
    \( j \leftarrow i_1 \), \( \ell \leftarrow i_2 \)\;
  \Repeat{There are no more states left to add}
  {
    \uIf{ \( P_{j \ell} = P_{\ell j} \)}{Move to the next step.}
    \uElseIf{ \( P_{j \ell }< P_{\ell j} \).} {Multiply \emph{row} \( \ell \)
      by \( P_{j \ell}/P_{\ell j} \)\;
      Adjust \( P_{\ell \ell} \) to make
      the \( \ell \) row sum to \( 1 \).\;}
    \uElseIf{ \( P_{j \ell} > P_{\ell j} \)}{ Multiply
          \emph{column} \( \ell \) by \( P_{\ell j}/P_{j \ell} \)\;
          Adjust \( P_{\ell \ell} \) to make
          the \( \ell \) \emph{row} sum to \( 1 \).\;}
        \tcp*[h]{The new matrix \( P^{\star} \) will now have
          \( P_{j \ell}^{\star} = P_{\ell j}^ {\star}\).\;}
    Choose another state \( i_3 \notin S \) which has nonzero transition
    probabilities to a state in \( S \).\;
        \( j \leftarrow i_3 \)  \( \ell \notin S \) with \( P_{i_3
        \ell} \ne 0 \) \;
    \( S \leftarrow \set{i_1, i_2, i_3} \).\;
   }   
  \tcp*[h]{After \( k - 1 \) steps \( S = \set{1, \dots, k} \).\;}
  Let \( P^{\star} \) be the final matrix.\;
\end{algorithm}

\begin{theorem}
    Let \( P \) be a \( k \times k \) transition matrix to which the
    Algorithm is applied, resulting in \( P^{\star} \).  Then \( P \) is
    reversible if and only if \( P^{\star} \) is symmetric.
\end{theorem}

\begin{proof}
    \begin{enumerate}
        \item
            \( (\Rightarrow \))
            \begin{enumerate}
                \item
                    Assume \( P \) is reversible, then by the lemma, \(
                    P^{\star} \) is reversible.
                \item
                    Let \( \phi = (\phi_1, \dots, \phi_k) \) be the
                    stationary distribution for \( P^{\star} \).
                \item
                    Note that \( P^{\star} \) is formed so that \( P_{ij}^
                    {\star} = P_{ji}^{\star} \) for \emph{a particular
                    subcollection} of \( (i,j) \) pairs which will
                    include each of \( 1, 2, \dots, k \) \emph{somewhere}
                    among the subcollection of transitions \( (i,j) \).
                \item
                    Let \( (i,j) \) be one of the transition pairs in
                    the collection. Since \( P^{\star} \) is reversible,
                    the detailed balance equations \( \phi_i P_{ij}^{\star}
                    = \phi_j P_{ji}^{\star} \) must hold for the pair \(
                    (i,j) \).
                \item
                    Using the equality from the symmetry, \( \phi_i =
                    \phi_j \).
                \item
                    But all the states \( 1, \dots, k \) appear in the
                    set of transitions somewhere in the collection, so \(
                    \phi_1 = \phi_2 = \cdots = \phi_k \).
                \item
                    Now take an arbitrary pair \( (i,j) \).  Since \( P^
                    {\star} \) is reversible, the detailed balance holds
                    for all \( i,j \).  Hence \( \phi_i P_{ij}^{\star} =
                    \phi_j P_{ji}^{\star} \) for all \( i,j \).
                \item
                    Since \( \phi_i = \phi_j \), for all pairs \( (i,j) \)
                    then \( P_{ij}^{\star} = P_{ji}^{\star} \), so \( P^
                    {\star} \) is symmetric.
            \end{enumerate}
        \item
            \( (\Leftarrow) \)

            If \( P^{\star} \) is symmetric, then \( P^{\star} \) is
            reversible, so the by the lemma \( P \) is reversible.
    \end{enumerate}
\end{proof}

\begin{remark}
    The Algorithm chooses the smaller of \( 2 \) matrix entries to
    change the matrix \( P \).  Also corrections are made to the
    diagonal elements in order to ensure the rows sum to \( 1 \).  In
    fact, by examining the loop equation \eqref{eq:reversiblemarkovchain:loopprob},
    correction of the diagonal elements is not necessary and the
    transformed matrix \( P^{\star} \) no longer needs to be a
    transition matrix.  The proof shows the important issue is whether \(
    P^{\star} \) is symmetric or not.
\end{remark}

\begin{example}
    Let
    \[
        P =
        \begin{pmatrix}
            17/40       & 0     & 3/40  & 1/2 \\
            0   & 11/20 & 1/4   & 1/5 \\
            3/10        & 1/4   & 9/20  & 0 \\
            1/2 & 1/20  & 0     & 9/20
        \end{pmatrix}
        .
    \] To check for reversibility, transform \( P \) by column or row
    operations.  First, the zeroes of \( P \) are symmetric, and \( P_{14}
    = P_{41} \), so symmetry for states \( \set{1,4} \) already exists.
    Now make the \( (1,3) \) entry match the \( (3,1) \) entry without
    losing the \( (1,4) \) and \( (4,1) \) symmetry.  It is possible to
    multiply column \( 3 \) by \( (3/10)/(3/40) = 4 \) or row \( 3 \) by
    \( (3/40)/(3/10) = 1/4 \).  Make the latter choice to obtain a new
    matrix with row \( 3 \) equal to \( (3/40, 1/16, 9/80, 0) \).  This
    is not a transition probability matrix because the row no longer
    sums to \( 1 \), so change the diagonal element at \( (3,3) \) to \(
    1 - 3/40 - 1/16 = 69/80 \).  Now the transformed matrix is
    \[
        P^{(1)} =
        \begin{pmatrix}
            17/40       & 0     & 3/40  & 1/2 \\
            0   & 11/20 & 1/4   & 1/5 \\
            3/40        & 1/16  & 69/80 & 0 \\
            1/2 & 1/20  & 0     & 9/20
        \end{pmatrix}
        .
    \] Now the set of included states is \( S = \set{1,4,3} \), so state
    \( 2 \) needs to be included.  To preserve the values in \( (1,4) \)
    and \( (4,1) \), and \( (1,3) \) and \( (3,1) \), only row \( 2 \)
    or column \( 2 \) can be changed.  Choose to multiply column \( 2 \)
    by \( 4 \) to make entries \( (4,2) \) and \( (2,4) \) equal.  The
    result is
    \[
        P^{(2)} =
        \begin{pmatrix}
            17/40       & 0     & 3/40  & 1/2 \\
            0   & 11/5  & 1/4   & 1/5 \\
            3/40        & 1/4   & 69/80 & 0 \\
            1/2 & 1/5   & 0     & 9/20
        \end{pmatrix}
        .
    \] Adjust the entries in the diagonal positions in rows \( 2 \), \(
    3 \), \( 4 \) to make the matrix a transition probability matrix
    \[
        P^{(\star)} =
        \begin{pmatrix}
            17/40       & 0     & 3/40  & 1/2 \\
            0   & 11/20 & 1/4   & 1/5 \\
            3/40        & 1/4   & 27/40 & 0 \\
            1/2 & 1/5   & 0     & 3/10
        \end{pmatrix}
        .
    \] Now \( P \) is reversible if and only if \( P^{\star} \) is
    reversible.  But \( P^{\star} \) is symmetric so it is automatically
    reversible.  Using the detailed balance equations makes it easy to
    compute the stationary distribution of \( P \), see the exercises.

\end{example}
\subsection*{Examples of Reversible Chains}

\subsubsection*{Negative Drift Random Walk on the Non-negative Integers}

Consider a negative drift simple random walk, restricted to be
non-negative by a reflecting boundary.  That is, \( P_{01} = 1 \) and
otherwise \( P_{i,i+1} = p < 0.5 \), \( P_{i,i-1} = 1 - p > 0.5 \).  The
time reversibility equations are
\begin{align*}
    \pi_0       &= (1-p) \pi_1 \\
    p \pi_{i}   &= (1-p) \pi_{i+1}
\end{align*}
so \( \pi_1 = \pi_0/(1-p) \), \( \pi_2 = p\pi_0/(1-p)^{2} \) and in
general \( \pi_1 = p^{n-1} \pi_0/(1-p)^n \).  Since \( \sum_{\nu} \pi_{\nu}
= 1 \),
\[
    \pi_0 \left( 1 + \frac{1}{1-p} \sum_{\nu} \left( \frac{p}{1-p}
    \right)^{\nu} \right) = 1.
\] Because \( \frac{p}{1-p} < 1 \) the geometric series converges and
\begin{align*}
    \pi_0       &= \frac{1}{2}\cdot \frac{1-2p}{1-p} \\
    \pi_n       &= \left( \frac{1}{2} - p \right) \left( \frac{p}{1-p}
    \right)^n, n \ge 1.
\end{align*}

\subsubsection*{Random Walks on Weighted Graphs} A \emph{random walk on
a weighted graph}%
\index{random walk! on graphs}%
\index{Markov chain ! random walk on a graph}
is a general and common example of a reversible Markov chain.  Assume
every undirected edge between vertices \( i \) and \( j \) in a
connected graph has a weight \( w_{ij} = w_{ji} \).  The Markov chain is
irreducible because the graph with edge weights greater than \( 0 \) is
complete.  When at \( i \), the walker goes to \( j \) with probability
proportional to \( w_{ij} \) , so
\[
    P_{ij} = \frac{w_{ij}}{\sum_{\nu} w_{i\nu}}.
\] Let
\[
    s = \sum_{i,j} w_{ij}
\] be the sum of all weights and let
\[
    \pi_i = \frac{\sum_{\nu} w_{i\nu}}{s}.
\] The probability transition matrix is reversible because
\[
    \pi_i P_{ij} = \frac{\sum_{\nu} w_{i\nu}}{s} \cdot \frac{w_{ij}}{\sum_
    {\nu} w_{i\nu}} = \frac{w_{ij}}{s} = \frac{w_{ji}}{s} = \frac{\sum_{\nu}
    w_{j\nu}}{s} \cdot \frac{w_{ji}}{\sum_{\nu} w_{j\nu}} = \pi_j P_{ji}.
\]

It is not necessary to forbid self-edges, some edges \( w_{ii} \) may be
nonzero.  However, \( w_{ii} \) appears only once in the sum \( s \),
while the value \( w_{ij} \) appears twice, once each for \( i \) and \(
j \).

In the simple case with no self-edges and all nonzero weights equal to \(
1 \), the invariant distribution is
\[
    \pi_i = \frac{%
    \operatorname{degree}
    (i)}{2 \cdot (\text{number of edges})}.
\]

\subsubsection*{Ehrenfest Urn Model}

The physicist P. Ehrenfest proposed the following model for statistical
mechanics and kinetic theory.  The motivation is diffusion through a
membrane.  Two urns labeled \( A \) and \( B \) contain a total of \( N \)
balls.  In the Ehrenfest urn model a%
\index{Ehrenfest urn model}%
\index{Markov chain ! Ehrenfest urn model}%
\index{urn model}
ball is selected at random with all selections equally likely, and moved
from the urn it is in to the other urn.  The state at each time is the
number of balls in the urn \( A \), from \( 0 \) to \( N \).  Then the
transition probability matrix is
\[
    P =
    \begin{pmatrix}
        0       & 1     & 0     & 0     & \cdots        & 0     & 0 \\
        \frac{1}{N}     & 0     & 1-\frac{1}{N} & 0     & \cdots
        & 0     & 0 \\
        0       & \frac{2}{N}   & 0     & 1-\frac{2}{N} & \cdots
        & 0     & 0 \\
        \vdots  & \vdots        & \vdots        & \vdots        & \ddots&
        \vdots  & \vdots \\
        0       & 0     & 0     & 0     & \cdots        & 0     & 1/N \\
        0       & 0     & 0     & 0     & \cdots        & 1     & 0 \\
    \end{pmatrix}
    .
\] The balls fluctuate between the two containers with a drift from the
one with the larger number of balls to the one with the smaller numbers.

A stationary distribution for this Markov chain has entry \( \pi_i =
\binom{N}{i}/2^N \).  This is the binomial distribution on \( N \), so
that for large \( N \), this can be approximated with the normal
distribution.  This conclusion is plausible given the physical origin of
the Markov chain as a model for diffusion.  All states are accessible
and all states communicate so the chain has a stationary distribution.
The chain is periodic with period \( 2 \) so it is not regular.

To show the Markov chain is reversible requires
\begin{align*}
    \pi_0 P_{01}        &= \pi_1 P_{10}, \\
    \pi_i P_{i,i+1}     &= \pi_{i+1} P_{i+1,i}, \\
    \pi_i P_{i,i-1}     &= \pi_{i-1} P_{i-1,i}, \\
    \pi_N P_{N,N-11}    &= \pi_{N-1} P_{N-1,N}.  \\
\end{align*}
See the exercises.

\visual{Section Starter Question}{../../../../CommonInformation/Lessons/question_mark.png}
\section*{Section Ending Answer}

It should not be possible to distinguish whether the video is played
forward or in reverse simply by viewing the video and observing the
movement of the ball between the urns.  The distribution of the balls is
stationary, or in physical terms, in equilibrium.  That is, a ball is as
likely to move from urn A to urn B in forward time, or from urn B to urn
A in reverse time.  Observing the movement of balls between the urns
provides no clue about the time direction of the video.

\subsection*{Sources}

The definition of reversibility, the Proposition, and the Reversibility
Condition Theorem are adapted from \link{https://www.math.ucdavis.edu/~gravner/MAT135B/materials/ch16.pdf}
{Gravner}.

Remarks surrounding the Reversibility Condition Theorem are adapted from
\link{https://www.sjsu.edu/faculty/guangliang.chen/Math263/lec5imeReversibility.pdf}
{Chen} and \link{http://www.columbia.edu/~ks20/stochastic-I/stochastic-I-Time-Reversibility.pdf}
{Sigman}.

Identifying reversibility with the Kolmogorov loop condition is adapted
from \link{https://digitalcommons.lsu.edu/cgi/viewcontent.cgi?article=1472
&context=cosa}{Brill et al.}
\cite{brill18}.  The proof that the Kolmogorov Loop Criterion is
equivalent to reversibility is adapted from
\link{https://en.wikipedia.org/wiki/Kolmogorov\%27s_criterion}
{Kolmogorovs Criterion}

The example of the negative drift random walk on the non-negative
integers is adapted from \link{http://www.columbia.edu/~ks20/stochastic-I/stochastic-I-Time-Reversibility.pdf}
{Sigman}.

The example of the random walk on a connected graph is adapted from
\link{https://www.math.ucdavis.edu/~gravner/MAT135B/materials/ch16.pdf}{Gravner}.

The problem about the random walk of a king on a checkerboard is
inspired by a problem in \link{https://www.math.ucdavis.edu/~gravner/MAT135B/materials/ch16.pdf}
{Gravner}.

\nocite{}
\nocite{}

\hr

\visual{Algorithms, Scripts, Simulations}{../../../../CommonInformation/Lessons/computer.png}
\section*{Algorithms, Scripts, Simulations}

\subsection*{Algorithm}

\subsection*{Scripts}

%% \input{ _scripts}

\hr

\visual{Problems to Work}{../../../../CommonInformation/Lessons/solveproblems.png}
\section*{Problems to Work for Understanding}
\renewcommand{\theexerciseseries}{}
\renewcommand{\theexercise}{\arabic{exercise}}
\begin{exercises}
\begin{exercise}
    Compute the stationary distribution for
    \[
        P =
        \begin{pmatrix}
            17/40       & 0     & 3/40  & 1/2 \\
            0   & 11/20 & 1/4   & 1/5 \\
            3/10        & 1/4   & 9/20  & 0 \\
            1/2 & 1/20  & 0     & 9/20
        \end{pmatrix}
    \] using the detailed balance equations.
\end{exercise}
\begin{solution}
    Use the detailed balance to obtain \( \pi_1P_{14} = \pi_4P_{41} \)
    and \( \pi_1P_{13} = \pi_3P_{31} \) and \( \pi_3P_{32} = \pi_2P_{23}
    \) so \( \pi_1(1/2) = \pi_4(1/2) \) and \( \pi_1(3/40) = \pi_3(3/10)
    \) and \( \pi_3(1/4) = \pi_2(1/4) \).  Hence \( \pi_2 = \pi_3 = 3\pi_1
    = 3\pi_4 \).  Since the sum of the probabilities is \( 1 \), \( \pi_1
    = \pi_4 = 3/8 \) and \( \pi_2 = \pi_3 = 1/8 \).
\end{solution}

\begin{exercise}
    Consider the \( 3 \times 3 \) square lattice graph in Figure~%
    \ref{fig:reversiblemarkovchain:sqlattice}.  The graph has \( 9 \)
    vertices with \( 10 \) edges between nearest lattice neighbors and
    self-loops at each vertex.  If a vertex has \( n \) edges then the
    probability of moving to a neighboring edge or staying at the vertex
    is \( \frac{1}{n+1} \) uniformly.  Find the stationary distribution
    for the random walk on this graph.

    \begin{figure}
        \centering
        \begin{asy}
            size(5inches);

            real myfontsize = 12; real mylineskip = 1.2*myfontsize; pen
            mypen = fontsize(myfontsize, mylineskip); defaultpen(mypen);

            real marge=1mm; pair z1=(0, 2), z2=(1, 2), z3=(2, 2); pair
            z4=(0, 1), z5=(1, 1), z6=(2, 1); pair z7=(0, 0), z8=(1, 0),
            z9=(2, 0);

            transform r=scale(1.0);

            object state1=draw("1",ellipse,z1,marge), state2=draw("2",ellipse,z2,marge),
            state3=draw("3",ellipse,z3,marge), state4=draw("4",ellipse,z4,marge),
            state5=draw("5",ellipse,z5,marge), state6=draw("6",ellipse,z6,marge),
            state7=draw("7",ellipse,z7,marge), state8=draw("8",ellipse,z8,marge),
            state9=draw("9",ellipse,z9,marge);

            add(new void(picture pic, transform t) { draw(pic, point(state1,E,t)--point
            (state2,W,t)); draw(pic, point(state1,S,t)--point(state4,N,t));
            });

            add(new void(picture pic, transform t) { draw(pic, point(state2,E,t)--point
            (state3,W,t)); draw(pic, point(state2,S,t)--point(state5,N,t));
            });

            add(new void(picture pic, transform t) { draw(pic, point(state3,S,t)--point
            (state6,N,t)); });

            add(new void(picture pic, transform t) { draw(pic, point(state4,E,t)--point
            (state5,W,t)); draw(pic, point(state4,S,t)--point(state7,N,t));
            });

            add(new void(picture pic, transform t) { draw(pic, point(state5,E,t)--point
            (state6,W,t)); draw(pic, point(state5,S,t)--point(state8,N,t));
            });

            add(new void(picture pic, transform t) { draw(pic, point(state5,E,t)--point
            (state6,W,t)); draw(pic, point(state5,S,t)--point(state8,N,t));
            });

            add(new void(picture pic, transform t) { draw(pic, point(state6,S,t)--point
            (state9,N,t)); });

            add(new void(picture pic, transform t) { draw(pic, point(state7,E,t)--point
            (state8,W,t)); });

            add(new void(picture pic, transform t) { draw(pic, point(state8,E,t)--point
            (state9,W,t)); });
        \end{asy}
        \caption{A \( 3 \times 3 \) square lattice graph with uniform
        transition probabilities.}%
        \label{fig:reversiblemarkovchain:sqlattice}
    \end{figure}
\end{exercise}
\begin{solution}
    This Markov chain has a stationary distribution
    \[
        \pi = (\frac{1}{11}, \frac{4}{33}, \frac{1}{11}, \frac{4}{33},
        \frac{5}{33}, \frac{4}{33}, \frac{1}{11}, \frac{4}{33}, \frac{1}
        {11}).
    \]
\end{solution}

\begin{exercise}
    In the game of Checkers, a ``king'' can move from a black square to
    any adjacent black square on an \( 8 \times 8 \) square array
    colored alternately red and black.
    \begin{enumerate}[label=(\alph*)]
    \item
        Assuming the king starts at one of the two corner squares of the
        chessboard, compute the expected number of steps before it
        returns to the starting position.
    \item
        Now assume two kings start at opposite corner squares, and move
        independently (and may occupy the same square).  What is now the
        expected number of steps before they simultaneously occupy their
        starting positions again?
\end{enumerate}
\end{exercise}
\begin{solution}
    \begin{enumerate}[label=(\alph*)]
    \item
        The king makes a random walk on a graph with \( 32 \) vertices (the
        black squares) with degrees \( 1 \) on \( 2 \) corner squares, \(
        2 \) on \( 12 \) side squares, and \( 4 \) on \( 18 \) interior
        squares.  For a corner square \( i \), \( \pi_i = \frac{1}{1
        \cdot 2+ 2 \cdot 12 + 4 \cdot 18} = \frac{1}{98} \) so the
        answer is \( 98 \).
      \item
        Label one corner starting square as state \( 1 \) and the
        other corner starting square as state \( 32 \).  The
        stationary probability of occupying state \( 1 \) is \(
        \pi_{1} = \frac{1}{98} \) and the stationary probability of occupying state
        \( 32 \) is also \( \pi_{1} = \frac{1}{98} \).
        Since the walks are independent processes, the joint
        probability of occupying both states simultaneously is \(
        \pi_1 \pi_{32} = \frac{1}{98^{2}} \) and the exprected number of steps
        is \( 98^2 \).
\end{enumerate}
\end{solution}

\begin{exercise}
    Show that if the Markov chain has a state accesible to and from
    every other state in exactly \( 1 \)  step (i.e.\ the \( j_0 \) row
    row and \( j_{0} \) column of the
    transition matrix have nonzero entries), then for the Kolmogorov
    loop criterion it is sufficient to check loops of only three states.
    (However, it is possible that no such state exists.)
\end{exercise}
\begin{solution}
    Suppose the Markov chain has a state \( j_{\star} \) with \(
    P_{i j_{\star}} > 0 \) and \( P_{j_{\star} i} > 0 \) for \( 1 < i < k \).  The goal is to show
    the Markov chain is reversible if and only if
    \[
      P_{j_{\star}, j_1} P_{j_1, j_2} P_{j_2, j_{\star}} = P_{j_{\star}, j_2} P_{j_2,
        j_1} P_{j_1, j_{\star}}
    \]
    for all \( 1 < j_1, j_2 < k\).  Then
    \[
      P_{j_1 j_2} = \frac{P_{j_{\star} j_2}}{P_{j_{\star} j_1}} P_{j_2 j_1}
      \frac{P_{j_1 j_{\star}}}{P_{j_2 j_{\star}}}.
    \]
    This rewrtten equality uses the hypothesis that \(
    P_{i j_{\star}} > 0 \) and \( P_{j_{\star} i} > 0 \).
    Start with the the loop probability
    \[
      P_{j_0 j_1} P_{j_1 j_2} \cdots  P_{j_{\ell-1} j_\ell} P_{j_\ell j_0}
    \]
    and substitute the rewritten three step equality to obtain
    \[
      \frac{P_{j_{\star} j_1}}{P_{j_{\star} j_0}} P_{j_1 j_0}
      \frac{P_{j_0 j_{\star}}}{P_{j_1 j_{\star}}}
      \frac{P_{j_{\star} j_2}}{P_{j_{\star} j_1}} P_{j_2 j_1}
      \frac{P_{j_1 j_{\star}}}{P_{j_2 j_{\star}}}
      \cdots
      \frac{P_{j_{\star} j_\ell}}{P_{j_{\star} j_{\ell-1}}} P_{j_\ell j_{\ell-1}}
      \frac{P_{j_{\ell-1} j_{\star}}}{P_{j_\ell j_{\star}}}
      \frac{P_{j_{\star} j_0}}{P_{j_{\star} j_\ell}} P_{j_0 j_\ell}
      \frac{P_{j_\ell j_{\star}}}{P_{j_0 j_{\star}}}.
    \]
    Many terms cancel in a telescoping fashion, and reversing the order of
            multiplication gives the right side of the Kolmogorov loop
            criterion.
            
    Brill et al. say Kelly \emph{Reversibility and Stochastic
      Networks}, 1978, Cambridge University Press, Exercise 1.5.2, says that if there is a
    state which can be accessed \emph{from} every other state in exactly \(
    1 \) step, i.e.\ a column of the
    transition matrix has nonzero entries, then it is sufficient to
    check loops of only \( 3 \) states.  However, neither source
    provides the proof, so it is possible that the exercise is
    mistaken.  No counterexample is provided here, so the less
    restrictive criterion remains open here.
\end{solution}
\end{exercises}
\hr

\visual{Books}{../../../../CommonInformation/Lessons/books.png}
\section*{Reading Suggestion:}

\bibliography{../../../../CommonInformation/bibliography}

%   \begin{enumerate}
%     \item
%     \item
%     \item
%   \end{enumerate}

\hr

\visual{Links}{../../../../CommonInformation/Lessons/chainlink.png}
\section*{Outside Readings and Links:}
\begin{enumerate}
    \item
    \item
    \item
    \item
\end{enumerate}

\section*{\solutionsname} \loadSolutions

\hr

\mydisclaim \myfooter

Last modified:  \flastmod

\end{document}

%%% Local Variables:
%%% TeX-master: t
%%% TeX-master: t
%%% TeX-master: t
%%% End:
