\documentclass[12pt]{article}

\input{../../../../etc/macros}
%\input{../../../../etc/mzlatex_macros}
\input{../../../../etc/pdf_macros}

\bibliographystyle{plain}

\begin{document}

\myheader
\mytitle

\hr

\sectiontitle{Reversible Markov Chain}

\hr

\usefirefox

% \hr

% \visual{Study Tip}{../../../../CommonInformation/Lessons/studytip.png}
% \section*{Study Tip}

\hr

\visual{Rating}{../../../../CommonInformation/Lessons/rating.png}
\section*{Rating} %one of 
% Everyone: contains no mathematics.
% Student: contains scenes of mild algebra or calculus that may require guidance.
Mathematically Mature: may contain mathematics beyond calculus with proofs.
% Mathematicians Only: prolonged scenes of intense rigor.

\hr

\visual{Section Starter Question}{../../../../CommonInformation/Lessons/question_mark.png}
\section*{Section Starter Question}

Imagine a video is made of the Ehrenfest urn model started in its
stationary distribution.  Would it be possible to distinguish whether
the video is played forward or in reverse simply by viewing the video
and observing the movement of the ball between the urns?

\hr

\visual{Key Concepts}{../../../../CommonInformation/Lessons/keyconcepts.png}
\section*{Key Concepts}

\begin{enumerate}
  \item   A Markov chain with invariant distribution $\pi$ is reversible if
  and only if there is a probability distribution $\pi$ such that
  \[
    \pi_i P_{ij} = \pi_j P_{ji}
  \]
  for all $i$ and $j$ and $\pi$ is the unique stationary distribution.
  \item   The condition says
  \[
    \Prob{X_n =i, X_{n+1} = j} = \Prob{X_n =j, X_{n+1} = i}
  \]
  so the chain transitions from $i$ to $j$ as often as it transitions
  from $j$ to $i$.
  \item   Assume the Markov chain is irreducible and positive recurrent and
  there is a distribution $\pi$ satisfying the detailed balance equation
  \[
    \pi_i P_{ij} = \pi_j P_{ji}.
  \]
  Then the distribution $pi$ is the stationary distribution of the chain.
  and the chain is reversible.
\item   Let $P$ be a $k \times k$ transition matrix to which an
  Algorithm of row and column operations
  is applied, resulting in $P^{\star}$.  Then $P$ is reversible if and
  only if $\P^{\star}$ is symmetric.
  \end{enumerate}

\end{enumerate}

\hr

\visual{Vocabulary}{../../../../CommonInformation/Lessons/vocabulary.png}
\section*{Vocabulary}
\begin{enumerate}
  \item  The chain $X$ is \defn{reversible} if  $X$ is an irreducible and positive recurrent chain, started at its unique invariant
distribution $\pi$. Recall that this means that $X_0$ has distribution
$\pi$, and all other $X_n$ have distribution $\pi$ as well.
Now suppose that, for every $n$, $X_1, X_2, \dots , X_n$ have the same joint distribution as the time-reversal
$X_n, X_{n−1}, \dots , X_1$.
\item   The condition
  \[
    \pi_i P_{ij} = \pi_j P_{ji}
  \]
  is the \defn{detailed balance equation}.
\item  \defn{Kolmogorov's loop criterion} is
  \[
  P_{j_0 j_1} P_{j_1,j_2} \cdots P_{j_{k-1} j_k} P_{j_k, j_0} =
  P_{j_0 j_k} P_{j_k j_{k-1}} \cdots P_{j_2 j_1} P_{j_1, j_0}
\]
for every finite sequence of distinct state $j_0, j_1, \dots, j_k$.
\end{enumerate}

\section*{Notation}
\begin{enumerate}
\item $X$, $X_0$, $X_n$ -- Markov chain, starting state, general step
  of the chain.
\item $k$ -- Number of states in the Markov chain.
\item $\pi$, $\pi_i$ -- Stationary distribution and general element of
  the stationary distribution.
\item $i$, $j$, $i_{\ell}$ etc. -- arbitrary states of the Markov
  chain.
\item $P$, $P_{ij}$ -- Probability transition matrix and general
  element of probability transition matrix.
\item $P^{\star}$ -- the matrix resulting from a row (or column) multiplication operation on $P$.
\end{enumerate}
\hr

\visual{Mathematical Ideas}{../../../../CommonInformation/Lessons/mathematicalideas.png}
\section*{Mathematical Ideas}
\subsection*{Reversibility}

\subsubsection*{Detailed Balance Equations}

Assume $X$ is an irreducible and positive recurrent chain, started at its unique invariant
distribution $\pi$. Recall that this means that $X_0$ has distribution
$\pi$, and all other $X_n$ have distribution $\pi$ as well.
Now suppose that, for every $n$, $X_1, X_2, \dots , X_n$ have the same joint distribution as the time-reversal
$X_n, X_{n−1}, \dots , X_1$. Then the chain $X$ is \defn{reversible}\index{reversible}
and its invariant distribution $\pi$ is also said to be reversible. This means that a recorded simulation of a reversible
chain looks the same if the ``movie'' is run backwards.

\begin{proposition}
  If a Markov chain is started in $\pi$, then the time-reversed chain
  has the Markov property.
\end{proposition}

\begin{proof}
  \begin{align*}
    & \Prob{X_\ell = i \given X_{\ell+1} = j, X_{\ell+2} = i_{\ell+2}, \dots, X_n
      = i_n} \\
    &\qquad = \frac{\Prob{X_\ell = i,   X_{\ell+1} = j, X_{\ell+2} = i_{\ell+2},
      \dots, X_n}{ X_{\ell+1} = j, X_{\ell+2} = i_{\ell+2}, \dots, X_n} \\
    &\qquad = \frac{\pi_i P_{ij}P_{j \ell+2}\codts P_{i_{n-1} i_n}}{\pi_j
      P_{j \ell+2}\codts P_{i_{n-1} i_n}} \\
    &\qquad = \frac{\pi_i P_{ij}}{\pi_j}
  \end{align*}
  which depends only on $i$ and $j$.
\item FOr reversibility, the previous expression must be the same as
  the forward transition probability $\Prob{X_{\ell+1} = i \given X_{j =
      j}} = P_{ji}$
  \end{enumerate}
\end{proof}

\begin{remark}
  The definition is inconvenient because it requires the
stationary distribution $\pi$ in advance  to check if the chain is time-reversible.  The
following theorem  avoids having to know $\pi$ in advance and can even
help find $\pi$.

\end{remark}
\begin{theorem}[Reversibility Condition]
  A Markov chain with invariant distribution $\pi$ is reversible if
  and only if there is a probability distribution $\pi$ such that
  \[
    \pi_i P_{ij} = \pi_j P_{ji}
  \]
  for all $i$ and $j$ and $\pi$ is the unique stationary distribution.
\end{theorem}

\begin{remark}
  The condition
  \[
    \pi_i P_{ij} = \pi_j P_{ji}
  \]
  is the \defn{detailed balance equation}.
\end{remark}

\begin{proof}
  \begin{description}
  \item[$\Rightarrow$] The condition follows immediately from the
    proof of the Proposition.
  \item[$\Leftarrow$] From the proposition, the time-reversed chain is
    Markov.  If the original and the time-reversed chain both
    start at the same invariant distribution and satisfy the
    condition, then the transition probabilities are the same.  Then
    by definition the chain is reversible. 
  \end{description}
\end{proof}

\begin{remark}
  The condition says
  \[
    \Prob{X_n =i, X_{n+1} = j} = \Prob{X_n =j, X_{n+1} = i}
  \]
  so the chain transitions from $i$ to $j$ as often as it transitions
  from $j$ to $i$.
\end{remark}

\begin{remark}
  If there exist $2$ states such that $P_{ij} > 0$ but $P_{ji} = 0$
  then the detailed balance equation fails, since the limiting probabilities of all
entries of an ergodic chain are nonzero. 
  Then the chain is not reversible.
\end{remark}

\begin{theorem}
  Assume the Markov chain is irreducible and positive recurrent and
  there is a distribution $\pi$ satisfying the detailed balance equation
  \[
    \pi_i P_{ij} = \pi_j P_{ji}.
  \]
  \begin{enumerate}
  \item The distribution $pi$ is the stationary distribution of the chain.
  \item The chain is reversible.
  \end{enumerate}
\end{theorem}

\begin{proof}
  Using the detailed balance equations
  \[
    \sum_j \pi_j P_{ij} = \sum_j \pi_i P_{ij} = \pi_i \sum_j P_{ij} =
    \pi_i.
  \]
  Since the chains is irreducible and positive recurrent, then $\pi$
  must be the unique stationary distribution.
\end{proof}

\subsubsection*{Kolmogorov's loop criterion}

Say that a transition probability matrix is
\emph{reversible}\index{reversible matrix} if the
corresponding Markov chain is reversible.  From the definition an
irreducible and positive recurrent Markov chain is reversible if and
only if
\[
  P_{j_0 j_1} P_{j_1,j_2} \cdots P_{j_{\ell-1} j_\ell} P_{j_\ell, j_0} =
  P_{j_0 j_\ell} P_{j_\ell j_{\ell-1}} \cdots P_{j_2 j_1} P_{j_1, j_0}
\]
for every finite sequence of distinct state $j_0, j_1, \dots, j_\ell$.
This is called \defn{Kolmogorov's loop criterion}.\index{Kolmogorov's loop criterion}
In words, Kolmogorov’s loop criterion says that a Markov transition matrix is
reversible if and only for every loop of distinct states, the forward loop probability product
equals the backward loop probability product.

For a two-state Markov chain Kolmogorov's loop criterion is always
satisfied since $P_{12} P_{21} = P_{12} P_{21}$.  If the transition
matrix is symmetric, then $P_{ij} = P_{ji}$ for all $i$, $j$, so
Kolmogorov's loop criterion is always satisfied and the chain is
reversible.  

Kelly, 1978, [4] (Exercise 1.5.2) notes that
if there is a state which can be accessed from every other state in exactly 1 step
(i.e. a column of the transition matrix with no zero entries), then it is sufficient to
check loops of only three states . However, it is possible that no
such state exists.

Usually loop checking involves much computational work.  The obvious
reason is that he number of loops that need
to be checked grows very quickly with $n$ where $n$ is the number of states. 
The following Proposition counts the number of equations that must be checked in order to
apply Kolmogorov’s loop criterion.

\begin{proposition}
For an $k$ state Markov chain, with $k \ge 3$, the number of
equations that must be checked for reversibility by Kolmogorov’s method is
\[
  \sum_{\nu=3}^n \binom{n}{\nu} \frac{(\nu-1)!}{2}
\]
\end{proposition}

\begin{proof}
  \begin{enumerate}
  \item FOr a $3$-state Markov chain, only $1$ equation
    \[
      P_{12}P_{23}P_{31} = P_{31} P_{32} P_{21}
    \]
    needs checking, since no $2$ step loops need to be checked and any
    other length $3$ loop over the same states results in the same
    equation.
  \item FOr $n=4$, we must check each loop of $3$ states and the each
    loop of $4$ states.
  \item For $3$ state loops, choose any $3$ of $4$ state and there is
    one equation for each.  FOr the $4$ state loop, fix the starting
    state.  Then there are $3!$ orders for the other states.  However,
     the other side of the equation is just the reversed path, so
     there are only $\frac{3!}{2}$ paths involving $4$ states with the
     first state fixed.  In total, there are
     \[
       \binom{4}{3} + \binom{4}{4} \frac{(4-1)!}{2} = 7
     \]
     equations.
   \item The previous argument easily generalizes to larger values of $n$.
  \end{enumerate}
\end{proof}

\begin{table}
  \centering
  \begin{tabular}{ccccccccccc}
    $k$& 1&2 &3 &4 &5 &6 & 7 & 8 & 9 & 10 \
    equations & 0 & 0 & 1 & 7 & 37 & 197 & 1{,}172 & 8{,}018 & 62{,}814 & 556{,}014
  \end{tabular}
  \caption{Number of equation to be chacked for a Markov chain with
    $n$ states.}
  \label{tab:reversiblemarkovchains:loopeqns}
\end{table}

Next define operations to transform transition matrices in such a way as to
preserve their reversibility status (either reversible or
non-reversible). These
transformations will be useful in creating new reversible Markov chains from existing
ones, and for checking reversibility of Markov chains.
A row multiplication operation on row $i$ of a Markov transition
matrix is the multiplication of row $i$ by a positive constant that leaves the sum
of the non diagonal elements at most $1$, followed by an adjustment to $p_{ii}$ to make
the row sum exactly to $1$.
A column multiplication operation on column $j$ of a Markov transition matrix is the multiplication of column $j$ by a positive constant of allowable
size (so no row sums exceed 1) followed by adjustments to all diagonal entries to
make every row sum exactly 1.

\begin{lemma}
  A Markov chain matrix P maintains its reversibility status after
a row multiplication operation or a column multiplication operation.
\end{lemma}

\begin{proof}
  Let the $i$th row of $P$ correspond to state $i$. The Kolmogorov
  loop criterion states
that $P$ is reversible if and only if for all loops of distinct states, the forward loop probability
product equals the backward loop probability product. If a loop does not include
state $i$, then a multiplication row operation on row $i$ has no effect on
the forward
and backward loop products. Otherwise, note that state $i$ appears in the first
subscript of a forward loop probability if and only if it appears in the first subscript of a
backward loop probability. So the row operation will have an identical effect on
both sides of the loop product. A similar conclusion holds for column product
operations.
\end{proof}

\begin{remark}
  Let $P^{\star}$ be the matrix resulting from a row (or column) multiplication operation on $P$, then the limiting probabilities for $P^{\star}$ in the lemma
are generally different than the limiting probabilities for $P$. 
\end{remark}

If the matrix $P$ is an $n \times n$ probability transition matrix, then a sequence of at
most $n −1$ row or column multiplication operations will be sufficient to determine
whether or not $P$ is reversible or not.
\begin{enumerate}
\item 
\item Pick two nonzero symmetric positions in $P$, say $P_{i_1,i_2}$ and $P_{i_2,i_1}$. Let $S_1 =
\set{i_1, i_2}$. If $P_{i_1,i_2} = P_{i_2,i_1}$, move to the next step. Otherwise, assume $P_{i_1,i_2 }< P_{i_2,i_1}$.
Multiply row $i_2$ by $P_{i_1,i_2}/P_{i_2,i_1}$ and adjust $P_{i_2,i_2}$ to make the row sum to $1$. If
$P_{i_1,i_2} > P_{i_2,i_1}$, multiply column $i_2$ by $P_{i_2,i_1}/P_{i_1,i_2}$ and adjust all diagonal entries so
that the rows sum to $1$. The new matrix $P^{\star}$ will now have $P_{i_1,i_2} = P_{i_2,i_1}^{\star}$.
\item Choose another state $i_3$ which has nonzero transition probabilities with a state
in $S_1$. Make the appropriate row or column multiplication operation on row or
column $i_3$. Set $S_2 = \set{i_1, i_2, i_3}$.
\item  Repeat the previous with a new state until there are no states left to add. After $n −1$
steps we have $S_n = \set{i_1, \dots, i_n}. Let $P$^{\star}$ be the
final matrix.
\end{enumerate}

\begin{theorem}
  Let $P$ be a $k \times k$ transition matrix to which the Algorithm
  is applied, resulting in $P^{\star}$.  Then $P$ is reversible if and
  only if $\P^{\star}$ is symmetric.
\end{theorem}

\begin{proof}
  \begin{enumerate}
  \item $\Rightarrow$
    \begin{enumerate}
    \item Assume $P$ is reversible, then by the lemma, $P^{\star}$ is
      reversible.
    \item Let $\phi = (\phi_1, \dots, \phi_k)$ be the stationary
      distribution for $P^{\star}$.
    \item Note that $P^{\star}$ is formed so that $P_{ij}^{\star} =
      P_{ij}^{\star}$ for the set of transitions $(i,j)$ which
      includes each of the states $1, \dots, k$ among the $(i,j)$
      pairs.  Let $(i,j)$ be one of the transition pairs in the
      collection.
    \item Since $P^{\star}$ is reversible, the detailed balance
      equations  $\phi_i P_{ij}^{\star} = \phi_j
      P_{ji}^{\star}$ must hold for the pair $(i,j)$.
    \item But all the states $1, \dots, k$ appear in the set of
      transitions somewhere in the collection, so $\phi_1 = \phi_2 =
      \cdots = \phi_k$.
    \item Now take an arbitrary pair $(i,j)$.  Since $P^{\star}$ is
      reversibile, the detailed balance holds for all $i,j$.  Hence
      $\phi_i P_{ij}^{\star} = \phi_j P_{ji}^{\star}$ for all $i,j$.
    \item Since $\phi_i = \phi_j$, then $P_{ij}^{\star} =
      P_{ji}^{\star}$, so $P^{\star}$ is symmetric.
    \end{enumerate}
  \item $\Leftarrow$

    If $P^{\star}$ is symmetric , then $P^{\star}$ is reversible, so
    the by the lemma $P$ is reversible.
  \end{enumerate}
\end{proof}

\begin{remark}
  The Algorithm chooses the smaller of $2$ matrix entries in order to
  modify the matrix $P$.  Also corrections are made to the diagonal
  elements in order to ensure the rows sum to $1$.  In fact, this is
  not necessary and the transformed matrix $P^{\star}$ no longer needs
  to be a transition matrix.  Examining the proof shows that that
  important issue is whether $P^{\star}$ is symmetric or not.
\end{remark}

\begin{example}
  Let
  \[
    P =
    \begin{pmatrix}
      17/40 & 0 & 3/40 & 1/2 \\
      0     & 11/20 & 1/4 & 1/5 \\
      3/10  & 1/4   & 9/20 & 0 \\
      1/2   & 1/20  & 0    & 9/20
    \end{pmatrix}.
  \]
  To check for reversibility, transform $P$ by column or row
  operations.  First, the zeroes of $P$ are symmetric, and $P_{14} =
  P_{41}}$, so that symmetry for states $\set{1,4}$ already exists.
Now make the $(1,3)$ entry match the $(3,1)$ entry without losing the
$(1,4)$ and $(4,1)$ symmetry.  It is possible to multiply column $3$
by $(3/10)/(3/40) = 4$ of by $(3/40)/(3/10) = 1/4$.   Make the latter
choice to obtain a new matrix with row $3$ equal to $(3/40, 1/16,
9/80, 0)$.  This is not a transition probability matrix because the
row no longer sums to $1$, so change the diagonal element at $(3,3)$
to $1 - 3/40 - 1/16 = 69/80$.  Now the transformed matrix is
  \[
    P^{(1)} =
    \begin{pmatrix}
      17/40 & 0 & 3/40 & 1/2 \\
      0     & 11/20 & 1/4 & 1/5 \\
      3/40  & 1/16   & 69/80 & 0 \\
      1/2   & 1/20  & 0    & 9/20
    \end{pmatrix}.
  \]
Now the set of included states is $S = \set{1,4,3}$, so state $2$
needs to be included.  To preserve the values in $(1,4)$ and $(4,1)$,
and $(1,3)$ and $(3,1)$, only row $2$ or column $2$ can be changed.
CHoose to multiply column $2$ by $4$ to make entries $(4,2)$ and
$(2,4)$ equal.  The result is
  \[
    P^{(2)} =
    \begin{pmatrix}
      17/40 & 0 & 3/40 & 1/2 \\
      0     & 11/5 & 1/4 & 1/5 \\
      3/40  & 1/4   & 69/80 & 0 \\
      1/2   & 1/5  & 0    & 9/20
    \end{pmatrix}.
  \]
Adjust the entries in the diagonal positions in rows $2$, $3$, $4$ to
make the matrix a transition probability matrix
  \[
    P^{(\star)} =
    \begin{pmatrix}
      17/40 & 0 & 3/40 & 1/2 \\
      0     & 11/20 & 1/4 & 1/5 \\
      3/40  & 1/16   & 27/40 & 0 \\
      1/2   & 1/5  & 0    & 3/10
    \end{pmatrix}.
  \]
Now $P$ is reversible if and only if $P^{\star}$ is reversible.  But
$P^{\star}$ is symmetric so it is automatically reversible.  Using the
detailed balance equations makes it easy to compute the stationary
distribution of $P$, see the exercises.

\end{example}
\subsection*{Examples of Reversible Chains}

\subsubsection*{Negative Drift Random Walk on the Non-negative
  integers}

Consider a negative drift simple
random walk, restricted to be non-negative by a reflecting boundary.
That is,  $P_{01} = 1$ and otherwise $P_{i,i+1} = p <
0.5$, $P_{i,i−1} = 1 − p > 0.5$.  
The time reversibility equations are
\begin{align*}
  \pi_0 &= (1-p) \pi \\
  p \pi_{i} &= (1-p) \pi_{i+1}
\end{align*}
so $\pi_1 = \pi_0/(1-p)$, $\pi_2 = p\pi_0/(1-p)^{2}$ and in general
$\pi_1 = p^{n-1} \pi_0/(1-p)^n$.  Since $\sum_{nu} \pi_{\nu} = 1$,
\[
  \pi \left( 1 + \frac{1}{1-p} \sum_{\nu} \left( \frac{p}{1-p} )^{\nu}
    \right) = 1
  \]
  Since $\frac{p}{1-p} < 1$ the geometric series converges and
  \begin{align*}
    \pi_0 &= \frac{1}{2}\cdot \frac{1-2p}{1-p} \\
    \pi_n &= \left( \frac{1}{2} - p \right) \left(  \frac{p}}{(1-p}
            \right)^2, n \ge 1.
  \end{align*}
  
\subsubsection*{Random Walks on Weighted Graphs}
 A \emph{random walks on a weighted graph}\index{random walks on a weighted graph}
  is a general and common example of a reversible Markov
  chain. Assume that every undirected edge between vertices $i$ and $j$
in a complete graph has a weight $w_{ij} = w_{ji}$.  The Markov chain
is irreducible because the graph with edge weights greater than $0$ is complete. Edges with zero
weight are not.  When at $i$, the walker goes to $j$ with probability proportional
to $w_{ij}$ , so that
\[
  P_{ij} = \frac{w_{ij}}{\sum_{\nu} w_{i\nu}}.
\]
Let
\[
  s = sum_{i,j} w_{ij}
\]
be the sum of all weights and let
\[
  \pi_i = \frac{\sum_{\nu} w_{i\nu}}{s}.
\]
The  probability transition matrix is reversible because
\[
  \pi_i P_{ij} = \frac{\sum_{\nu} w_{i\nu}}{s} \cdot
  \frac{w_{ij}}{\sum_{\nu} w_{i\nu}} = \frac{w_{ij}}{s} =
  \frac{w_{ji}}{s} = \frac{\sum_{\nu} w_{j\nu}}{s} \cdot
  \frac{w_{ji}}{\sum_{\nu} w_{j\nu}} = \pi_j P_{ji}.
\]

It is not necessary to forbid self-edges, some edges $w_{ii}$ may be
nonzero.  However, $w_{ii}$ appears only once in the sum $s$, while
the value $w_{ij}$ appears twice, once each for $i$ and $j$.

In the simple case with no self-edges and all nonzero weights equal
to $1$, the invariant distribution is
\[
  \pi_i = \frac{\operatorname{degree}(i)}{2 \cdot (\text{number of
      edges})}.
\]


\subsubsection*{Ehrenfest Urn Model}

The physicist P. Ehrenfest proposed the following model for statistical
mechanics and kinetic theory.  The motivation is diffusion through a
membrane.  Two urns labeled \( A \) and \( B \) contain a total of \( N \)
balls.  In the \defn{Ehrenfest urn model} a%
\index{Ehrenfest urn model}
\index{Markov chain ! Ehrenfest urn model}
\index{urn model}
ball is selected at random with all selections equally likely, and moved
from the urn it is in to the other urn.  The state at each time is the
number of balls in the urn \( A \), from \( 0 \) to \( N \).  Then the
transition probability matrix is
\[
    P =
    \begin{pmatrix}
        0 & 1 & 0 & 0 & \cdots & 0 & 0 \\
        \frac{1}{N} & 0 & 1-\frac{1}{N} & 0 & \cdots & 0 & 0 \\
        0 & \frac{2}{N} & 0 & 1-\frac{2}{N} & \cdots & 0 & 0 \\
        \vdots & \vdots & \vdots & \vdots & \ddots& \vdots & \vdots \\
        0 & 0 & 0 & 0 & \cdots & 0 & 1/N \\
        0 & 0 & 0 & 0 & \cdots & 1 & 0 \\
    \end{pmatrix}
    .
\] The balls fluctuate between the two containers with a drift from the
one with the larger number of balls to the one with the smaller numbers.

A stationary distribution for this Markov chain has entry \( \pi_i =
\binom{N}{i}/2^N \).   This is the binomial distribution on \( N \),
so that for large \( N \), this can be approximated with the normal
distribution.  This conclusion is plausible given the physical origin of
the Markov chain as a model for diffusion. This Markov chain is
ergodic.  All states are accessible and all states communicate.  The
chain is periodic with period \( 2 \) so it is not regular.

To show that the Markov chain is reversible requires
\begin{align*}
  \pi_0 P_{01} \= \pi_1 P_{10}, \\
  \pi_i P_{i,i+1} \= \pi_{i+1} P_{i+1,i}, \\
  \pi_i P_{i,i-1} \= \pi_{i-1} P_{i-1,i}, \\
  \pi_N P_{N,N-11} \= \pi_{N-1} P_{N-1,N}. \\
\end{align*}
See the exercises.

\visual{Section Starter Question}{../../../../CommonInformation/Lessons/question_mark.png}
\section*{Section Ending Answer}

It should not be possible to distinguish whether
the video is played forward or in reverse simply by viewing the video
and observing the movement of the ball between the urns.  The
distribution of the balls is stationary, or in physical terms, in
equilibrium.  That is, a ball is as likely to move from urn A to
urn B in forward time, or from urn B to urn A in reverse time.
Observing the movement of balls between the urns provides no clue
about the time direction of the video.

\subsection*{Sources}

The definition of reversibility, the Proposition,  and the
Reversibility Condition Theorem are adapted from
\link{https://www.math.ucdavis.edu/~gravner/MAT135B/materials/ch16.pdf}{Gravner}.

Remarks surrounding the Reversibility Condition Theorem are
adapted from \link{https://www.sjsu.edu/faculty/guangliang.chen/Math263/lec5imeReversibility.pdf}{Chen} and
  \link{http://www.columbia.edu/~ks20/stochastic-I/stochastic-I-Time-Reversibility.pdf}{Sigman}.


Identifying reversibility with the Kolmogorov loop condition is
adapted from
\link{https://digitalcommons.lsu.edu/cgi/viewcontent.cgi?article=1472&context=cosa}{Brill
et al.}, \cite{brill18}.

The example of the negative drift random walk on the non-negative integers is adapted
from 
  \link{http://www.columbia.edu/~ks20/stochastic-I/stochastic-I-Time-Reversibility.pdf}{Sigman}.

The example of the random walk on a connected graph is adapted from 
\link{https://www.math.ucdavis.edu/~gravner/MAT135B/materials/ch16.pdf}{Gravner}.

The problem about the random walk of a king on a checkerboard is
inspired by a problem in 
\link{https://www.math.ucdavis.edu/~gravner/MAT135B/materials/ch16.pdf}{Gravner}.

\nocite{}
\nocite{}

\hr

\visual{Algorithms, Scripts, Simulations}{../../../../CommonInformation/Lessons/computer.png}
\section*{Algorithms, Scripts, Simulations}

\subsection*{Algorithm}

\subsection*{Scripts}

\input{ _scripts}

\hr

\visual{Problems to Work}{../../../../CommonInformation/Lessons/solveproblems.png}
\section*{Problems to Work for Understanding}
\renewcommand{\theexerciseseries}{}
\renewcommand{\theexercise}{\arabic{exercise}}

\begin{exercise}
  Compute the stationary distribution for
  \[
    P =
    \begin{pmatrix}
      17/40 & 0 & 3/40 & 1/2 \\
      0     & 11/20 & 1/4 & 1/5 \\
      3/10  & 1/4   & 9/20 & 0 \\
      1/2   & 1/20  & 0    & 9/20
    \end{pmatrix}
  \]
  using the detailed balance equations.
\end{exercise}
\begin{solution}
  Use the detailed balance to obtain $\pi_1P_{14} = \pi_4P_{41}$ and $\pi_1P_{13} = \pi_3P_{31}$
and $\pi_3P_{32} = \pi_2P_{23}$ so $\pi_1(1/2) = \pi_4(1/2)$ and $\pi_1(3/40) = \pi_3(3/10)$ and $\pi_3(1/4) = \pi_2(1/4)$.
Hence $\pi_2 = \pi_3 = 3\pi_1 = 3\pi_4$. Since the sum of the probabilities is $1$,
$\pi_1 = \pi_4 = 3/8$ and $\pi_2 = \pi_3 = 1/8$.
\end{solution}

\begin{exercise}
    Consider the $3 \times 3$ square lattice graph in
  Figure~\ref{fig:reversiblemarkovchain:sqlattice}. The graph has $9$
  vertices with $10$ edges between nearest lattice neighbors and
  self-loops at each vertex.  If a
  vertex has $n$ edges then the probability of moving to a neighboring
  edge or staying at the vertex is  $\frac{1}{n+1}$ uniformly.
  Find the stationary distribution for the random walk on this graph.
  
  \begin{figure}
  \centering
  \begin{asy}
size(5inches);

real myfontsize = 12;
real mylineskip = 1.2*myfontsize;
pen mypen = fontsize(myfontsize, mylineskip);
defaultpen(mypen);

real marge=1mm;
pair z1=(0, 2), z2=(1, 2), z3=(2, 2);
pair z4=(0, 1), z5=(1, 1), z6=(2, 1);
pair z7=(0, 0), z8=(1, 0), z9=(2, 0);

transform r=scale(1.0);

object state1=draw("1",ellipse,z1,marge),
state2=draw("2",ellipse,z2,marge),
state3=draw("3",ellipse,z3,marge),
state4=draw("4",ellipse,z4,marge),
state5=draw("5",ellipse,z5,marge),
state6=draw("6",ellipse,z6,marge),
state7=draw("7",ellipse,z7,marge),
state8=draw("8",ellipse,z8,marge),
state9=draw("9",ellipse,z9,marge);

add(new void(picture pic, transform t) {
    draw(pic, point(state1,E,t)--point(state2,W,t));
    draw(pic, point(state1,S,t)--point(state4,N,t));
});

add(new void(picture pic, transform t) {
    draw(pic, point(state2,E,t)--point(state3,W,t));
    draw(pic, point(state2,S,t)--point(state5,N,t));
});

add(new void(picture pic, transform t) {
    draw(pic, point(state3,S,t)--point(state6,N,t));
});

add(new void(picture pic, transform t) {
    draw(pic, point(state4,E,t)--point(state5,W,t));
    draw(pic, point(state4,S,t)--point(state7,N,t));
});

add(new void(picture pic, transform t) {
    draw(pic, point(state5,E,t)--point(state6,W,t));
    draw(pic, point(state5,S,t)--point(state8,N,t));
});

add(new void(picture pic, transform t) {
    draw(pic, point(state5,E,t)--point(state6,W,t));
    draw(pic, point(state5,S,t)--point(state8,N,t));
});

add(new void(picture pic, transform t) {
    draw(pic, point(state6,S,t)--point(state9,N,t));
});

add(new void(picture pic, transform t) {
    draw(pic, point(state7,E,t)--point(state8,W,t));
});

add(new void(picture pic, transform t) {
    draw(pic, point(state8,E,t)--point(state9,W,t));
});
\end{asy}
  \caption{A $3 \times 3$ square lattice graph with uniform transition
    probabilities.}
  \label{fig:standardexamples:sqlattice}
\end{figure}
\end{exercise}
\begin{solution}
    This Markov chain has a stationary distribution
  \[ \pi = (\frac{1}{11}, \frac{4}{33}, \frac{1}{11}, \frac{4}{33},
    \frac{5}{33}, \frac{4}{33}, \frac{1}{11}, \frac{4}{33}, \frac{1}{11}).
  \]
\end{solution}

\begin{exercise}
  In the game of Checkers, a ``king'' can move from a black
  square to any adjacent black square on an $8 \times 8$ square array
  colored alternately red and black.
    \begin{enumerate}[label=(\alpha*)]
  \item Assuming
that the king starts at one of the four corner squares of the chessboard, compute the expected
number of steps before it returns to the starting position. 
\item  Now assume two kings 
start at opposite corner squares, and move independently (and may
occupy the same square). What is now the expected number of
steps before they simultaneously occupy their starting positions? 
  \end{enumerate}
\end{exercise}
\begin{solution}
  \begin{enumerate}[label=(\alpha*)]
  \item The king makes a random walk on a graph with $32$ vertices (the
    black squares) and degrees $1$ on $2$ corner squares), $2$ on
 $12$ side squares), and $4$ on $18$ interior squares. For a corner
 square $i$, $\pi_i = \frac{1}{1 \cdot 2+ 2 \cdot 12 + 4 \cdot 18} = \frac{1}{98}$
so the answer is $98$.
\item Since the walks are independent processes, the number of steps
  is $98$.
  \end{enumerate}
\end{solution}

\hr

\visual{Books}{../../../../CommonInformation/Lessons/books.png}
\section*{Reading Suggestion:}

\bibliography{../../../../CommonInformation/bibliography}

%   \begin{enumerate}
%     \item 
%     \item 
%     \item 
%   \end{enumerate}

\hr

\visual{Links}{../../../../CommonInformation/Lessons/chainlink.png}
\section*{Outside Readings and Links:}
\begin{enumerate}
  \item  
  \item  
  \item  
  \item 
\end{enumerate}

\section*{\solutionsname}
\loadSolutions

\hr

\mydisclaim \myfooter

Last modified:  \flastmod

\end{document}

%%% Local Variables:
%%% mode: latex
%%% TeX-master: t
%%% End:
